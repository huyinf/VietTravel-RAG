\documentclass{article}
\usepackage[vietnamese]{babel}
\usepackage[letterpaper,top=1cm,bottom=1cm,left=1.5cm,right=1.5cm,marginparwidth=1.75cm]{geometry}
\usepackage{amsmath}
\usepackage{graphicx}
\usepackage[colorlinks=true, allcolors=blue]{hyperref}
\title{Cẩm nang du lịch Thành Phố Hồ Chí Minh}

\begin{document}
\begin{center}
    \fontsize{18}{20}\textbf{Cẩm nang du lịch Thành Phố Hồ Chí Minh}
\end{center}
\begin{abstract}
    (Giới thiệu)
\end{abstract}
\section*{Điểm du lịch}

\section{Chợ Bến Thành}
\begin{itemize}
    \item{\textbf{Địa chỉ}} Đường Lê Lợi, phường Bến Thành, quận 1.
    \item{\textbf{Giờ mở cửa:}} 7h - 20h.
    \item{\textbf{Giá vé/ chi phí:}}
    \item{\textbf{Phân loại điểm du lịch}}
    \item{\textbf{Nguồn thông tin}}
    \item{\textbf{Thông tin về địa điểm:}} Đây là một trong những địa điểm du lịch Thành phố Hồ Chí Minh nổi tiếng nhất nằm ngay trung tâm thành phố và luôn tấp nập người ra vào mua bán, du khách check in mỗi ngày. Chợ Bến Thành buôn bán tất tần tật những mặt hàng từ thực phẩm, vải vóc, quần áo, đồ gia dụng, đặc sản Sài Gòn,… và quan trọng là khu ẩm thực cực kỳ hấp dẫn ở chợ và những góc check in ấn tượng sẽ làm bạn luôn muốn lưu lại nơi đây thật lâu.
\end{itemize}

\begin{itemize}
    \item{\textbf{Các thông tin khác - Cách di chuyển}}
\end{itemize}

\section{Dinh Độc Lập}
\begin{itemize}
    \item{\textbf{Địa chỉ}} 135 Nam Kỳ Khởi Nghĩa, Bến Nghé, Quận 1.
    \item{\textbf{Giờ mở cửa:}} 7h30 – 11h; 13h30 – 17h.
    \item{\textbf{Giá vé/ chi phí:}} 10.000đ – 40.000đ.
    \item{\textbf{Phân loại điểm du lịch}}
    \item{\textbf{Nguồn thông tin}}
    \item{\textbf{Thông tin về địa điểm:}} Dinh Độc Lập hay còn gọi là hội trường Thống Nhất, đây là một công trình được xây dựng bởi người Pháp, từ thời Pháp thuộc. Đối với người dân Sài Gòn, Dinh Độc Lập là một di tích lịch sử mang ý nghĩa hòa bình và toàn vẹn lãnh thổ. Nơi đây đã được công nhận là 1 trong 10 di tích quốc gia đặc biệt của Việt Nam vào năm 2009. Dinh Độc Lập là nơi trưng bày và lưu giữ những hình ảnh, hiện vật giá trị từ những năm của thế kỷ 19 theo từng chủ đề khác nhau. Đặc biệt nơi đây vẫn còn giữ nguyên nội thất của những căn phòng quan trọng như phòng khánh tiết, phòng nội các,... Nếu bạn là một du khách yêu lịch sử, chắc hẳn Dinh Độc Lập sẽ là một trong những điểm nên đến ở Sài Gòn được yêu thích trong chuyến đi của bạn.
\end{itemize}

\begin{itemize}
    \item{\textbf{Các thông tin khác - Cách di chuyển}}
\end{itemize}

\section{Nhà thờ Đức Bà}
\begin{itemize}
    \item{\textbf{Địa chỉ}} số 01 Công Xã Paris, Bến Nghé, Quận 1.
    \item{\textbf{Giờ mở cửa:}} 6h – 20h.
    \item{\textbf{Giá vé/ chi phí:}}
    \item{\textbf{Phân loại điểm du lịch}}
    \item{\textbf{Nguồn thông tin}}
    \item{\textbf{Thông tin về địa điểm:}} Nhà thờ Đức Bà cũng nằm ngay trung tâm quận 1, nhà thờ Đức Bà là địa điểm du lịch thành phố Hồ Chí Minh mà bạn nhất định phải ghé đến. Nhà thờ Đức Bà là một công trình kiến trúc độc đáo mang đậm phong cách Châu Âu, là nơi sinh hoạt và tổ chức các buổi Thánh lễ cho những người theo đạo Công giáo ở Sài Thành. Ghé đến tham quan nhà thờ Đức Bà, bạn có thể đến vào buổi sáng để có thể ngắm được trọn vẹn vẻ đẹp của nhà thờ dưới ánh sáng ban ngày, check in ở mọi ngóc ngách trong nhà thờ đậm chất Gothic,… đặc biệt nhà thờ luôn đón chào những du khách, người dân và bất kỳ ai đến đây để nghe giảng đạo.
\end{itemize}

\begin{itemize}
    \item{\textbf{Các thông tin khác - Cách di chuyển}}
\end{itemize}

\section{Bảo tàng lịch sử Việt Nam}
\begin{itemize}
    \item{\textbf{Địa chỉ}} số 2 Nguyễn Bỉnh Khiêm, Quận 1.
    \item{\textbf{Giờ mở cửa:}} 8h – 17h.
    \item{\textbf{Giá vé/ chi phí:}} 2.000đ – 15.000đ.
    \item{\textbf{Phân loại điểm du lịch}}
    \item{\textbf{Nguồn thông tin}}
    \item{\textbf{Thông tin về địa điểm:}} Bảo tàng lịch sử Việt Nam được xây dựng và hoạt động từ những năm đầu thế kỷ 20, là nơi lưu giữ và bảo tồn những hình ảnh, cổ vật từ thuở sơ khai đến nay. Bảo tàng lịch sử Việt Nam thu hút phần lớn những du khách yêu lịch sử và kiến trúc pha trộn giữa 2 phong cách Á – Âu độc đáo. Ngoài là nơi lưu giữ nét văn hoa truyền thống của đất nước, bảo tàng lịch sử Việt Nam còn là một trong những điểm du lịch thành phố Hồ Chí Minh ấn tượng với những góc check in đẹp.
\end{itemize}

\begin{itemize}
    \item{\textbf{Các thông tin khác - Cách di chuyển}}
\end{itemize}

\section{Địa đạo Củ Chi}
\begin{itemize}
    \item{\textbf{Địa chỉ}} TL15, Phú Hiệp, Củ Chi.
    \item{\textbf{Giờ mở cửa:}} 8h – 17h.
    \item{\textbf{Giá vé/ chi phí:}} 20.000đ – 40.000đ.
    \item{\textbf{Phân loại điểm du lịch}}
    \item{\textbf{Nguồn thông tin}}
    \item{\textbf{Thông tin về địa điểm:}} Địa đạo Củ Chi là di tích lịch sử nằm ở ngoại ô, cách trung tâm Tp.HCM khoảng 70km. Đây là một trong những điểm nên đến ở Sài Gòn được yêu thích, đặc biệt là đối với các du khách nước ngoài. Nơi đây là di tích quốc gia đặc biệt, gắn liền với lịch sử chiến tranh Việt Nam, từng là căn cứ địa quan trọng và cũng là nơi sinh hoạt chung bao gồm bệnh xá, nhà bếp, nơi làm việc,… của quân đội ta trong cuộc kháng chiến chống quân xâm lược, giành lại toàn vẹn lãnh thổ.
\end{itemize}

\begin{itemize}
    \item{\textbf{Các thông tin khác - Cách di chuyển}}
\end{itemize}

\section{Phố đi bộ Nguyễn Huệ}
\begin{itemize}
    \item{\textbf{Địa chỉ}: Đường Nguyễn Huệ, quận 1.}
    \item{\textbf{Giờ mở cửa:} Cả ngày.}
    \item{\textbf{Giá vé/ chi phí:}}
    \item{\textbf{Phân loại điểm du lịch} }
    \item{\textbf{Nguồn thông tin}}
    \item{\textbf{Thông tin về địa điểm:}}
\end{itemize}
Phố đi bộ Nguyễn Huệ từ lâu đã là một điểm đến thú vị ở Sài Gòn đối với người dân bản địa và cả các du khách. Đây là một trong những địa điểm vui chơi ở Quận 1 được nhiều người ghé tới đặc biệt vào mỗi tối và dịp cuối tuần. Ở phố đi bộ Nguyễn Huệ, bạn sẽ có cơ hội tham gia vào các hoạt động như trình diễn thời trang, nghệ thuật đường phố, vui chơi giải trí, ẩm thực,... rất đặc sắc. Nơi đây là điểm check in hot nhất nhì Sài Thành mà bạn không thể bỏ lỡ trong chuyến du lịch Sài Gòn của mình.
\begin{itemize}
    \item{\textbf{Các thông tin khác - Cách di chuyển}}
\end{itemize}

\section{Thảo Cầm Viên}
\begin{itemize}
    \item{\textbf{Địa chỉ}: 2 Nguyễn Bỉnh Khiêm, Bến Nghé, Quận 1.}
    \item{\textbf{Giờ mở cửa:} 7h – 18h30.}
    \item{\textbf{Giá vé/ chi phí:} 30.000 - 50.000đ.}
    \item{\textbf{Phân loại điểm du lịch} }
    \item{\textbf{Nguồn thông tin}}
    \item{\textbf{Thông tin về địa điểm:}}
\end{itemize}
Một địa điểm du lịch thành phố Hồ Chí Minh phù hợp cho mọi lứa tuổi đó là Thảo Cầm Viên. Đây là khu bảo tồn động thực vật lớn ở trung tâm thành phố, có không gian rộng rãi với những khu vui chơi giải trí, khu nuôi dưỡng động vật hoang dã, khu trồng và chăm sóc thực vật,...rất thoáng mát. Thảo Cầm Viên là một trong những điểm nên đến ở Sài Gòn thích hợp để bạn khám phá các loại động thực vật quý hiếm và thư giãn sau những ồn ào ở thành phố.
\begin{itemize}
    \item{\textbf{Các thông tin khác - Cách di chuyển}}
\end{itemize}

\section{Bảo tàng tranh 3D Artinus}
\begin{itemize}
    \item{\textbf{Địa chỉ}: số 2 Khu Him Lam, quận 7.}
    \item{\textbf{Giờ mở cửa:} 9h - 22h.}
    \item{\textbf{Giá vé/ chi phí:} 200.000 - 300.000đ.}
    \item{\textbf{Phân loại điểm du lịch} }
    \item{\textbf{Nguồn thông tin}}
    \item{\textbf{Thông tin về địa điểm:}}
\end{itemize}
Là một bảo tàng tranh 3D lớn nhất Sài Gòn, Artinus là nơi check in cực hấp dẫn với nhiều du khách và giới trẻ Sài Thành. Bảo tàng tranh 3D Artinus có không gian kích thích sự sáng tạo ở mọi lứa tuổi qua từng khung ảnh. Bạn sẽ cảm thấy mình thật bé nhỏ khi đứng giữa những bức tranh 3D ấn tượng trong bảo tàng. Bên trong bảo tàng Artinus, những bức tranh 3D được sắp xếp theo 9 chủ đề khác nhau như Đại Dương, Việt Nam, Ai Cập, Tình Yêu, Thần Tiên,...
\begin{itemize}
    \item{\textbf{Các thông tin khác - Cách di chuyển}}
\end{itemize}

\section{Tòa nhà Bitexco}
\begin{itemize}
    \item{\textbf{Địa chỉ}: Số 19-25 Nguyễn Huệ, Phường Bến Nghé, Quận 1.}
    \item{\textbf{Giờ mở cửa:} 9h - 22h.}
    \item{\textbf{Giá vé/ chi phí:}}
    \item{\textbf{Phân loại điểm du lịch} }
    \item{\textbf{Nguồn thông tin}}
    \item{\textbf{Thông tin về địa điểm:}}
\end{itemize}
Tòa nhà Bitexco hay còn gọi là Tháp tài chính Bitexco, đây là một trong những tòa nhà cao nhất ở thành phố Hồ Chí Minh từ những năm 2010, sau này đã có tòa tháp khác cao hơn như Landmark 81 nhưng Bitexco mãi là một trong những biểu tượng của Sài Thành, đánh dấu cho sự phát triển mới của thành phố mang tên Bác.
\begin{itemize}
    \item{\textbf{Các thông tin khác - Cách di chuyển}}
\end{itemize}

\section{Công viên nước Đầm Sen}
\begin{itemize}
    \item{\textbf{Địa chỉ}: Số 3 Hòa Bình, phường 3, quận 11.}
    \item{\textbf{Giờ mở cửa:} 9h - 18h.}
    \item{\textbf{Giá vé/ chi phí:} 80.000đ - 150.000đ.}
    \item{\textbf{Phân loại điểm du lịch} }
    \item{\textbf{Nguồn thông tin}}
    \item{\textbf{Thông tin về địa điểm:}}
\end{itemize}
Nhắc đến du lịch Sài Gòn tất nhiên cũng phải nhắc đến công viên nước Đầm Sen - một trong các khu du lịch ở Sài Gòn được yêu thích dành cho mọi lứa tuổi. Công viên nước Đầm Sen có diện tích lên đến 3000m2, là một trong những điểm nên đến ở Sài Gòn thu hút rất nhiều người dân và du khách. Nơi đây được chia thành 2 khu công viên văn hóa là nơi tổ chức các sự kiện văn hóa, quảng bá, nhạc kịch,... và khu công viên nước là nơi có rất nhiều trò chơi hấp dẫn như tàu lượn siêu tốc, trượt nước, hồ bơi, nhà phao,...
\begin{itemize}
    \item{\textbf{Các thông tin khác - Cách di chuyển}}
\end{itemize}

\section{Chợ Bến Thành}
\begin{itemize}
    \item{\textbf{Địa chỉ}} Phường Bến Thành, Quận 1, Thành phố Hồ Chí Minh
    \item{\textbf{Giờ mở cửa:}} 7h30 - 18h
    \item{\textbf{Giá vé/ chi phí:}} Miễn phí
    \item{\textbf{Phân loại điểm du lịch} }
    \item{\textbf{Nguồn thông tin}}
    \item{\textbf{Thông tin về địa điểm:}}
\end{itemize}
Chợ Bến Thành là một trong những điểm đến hấp dẫn không thể bỏ qua khi bạn đến TPHCM. Với hơn 3000 gian hàng, và là nơi quy tụ rất nhiều mặt hàng từ quần áo, giày dép, túi xách cho đến đồ lưu niệm và thực phẩm đặc sản. Bạn có thể tìm thấy những chiếc áo dài truyền thống đẹp mắt hoặc những món đồ handmade độc đáo. Ngoài ra, Chợ Bến Thành cũng nổi tiếng với những món ăn đường phố hấp dẫn như bánh xèo, bánh cuốn, hay hủ tiếu Nam Vang. Đây là nơi tuyệt vời để khám phá văn hóa và ẩm thực của người Sài Gòn.
\begin{itemize}
    \item{\textbf{Các thông tin khác - Cách di chuyển}}
\end{itemize}

\section{Công viên Tao Đàn}
\begin{itemize}
    \item{\textbf{Địa chỉ}} Phường Bến Thành, Quận 1, Thành phố Hồ Chí Minh
    \item{\textbf{Giờ mở cửa:}} 7h - 22h
    \item{\textbf{Giá vé/ chi phí:}} Miễn phí
    \item{\textbf{Phân loại điểm du lịch} }
    \item{\textbf{Nguồn thông tin}}
    \item{\textbf{Thông tin về địa điểm:}}
\end{itemize}
Với diện tích rộng lớn và cây xanh um tùm, Công viên Tao Đàn là điểm đến lý tưởng để thoát khỏi sự ồn ào và náo nhiệt của thành phố. Công viên này có nhiều tiện ích như sân chơi trẻ em, khu vực tập thể dục ngoài trời và đường chạy bộ xung quanh ao cá. Bạn có thể dạo chơi trong không gian yên bình của công viên, ngắm nhìn hoa lá và hít thở không khí trong lành. Nếu bạn muốn tìm một nơi để thư giãn sau những ngày làm việc căng thẳng, Công viên Tao Đàn là lựa chọn hoàn hảo.
\begin{itemize}
    \item{\textbf{Các thông tin khác - Cách di chuyển}}
\end{itemize}

\section{Nhà thờ Đức Bà}
\begin{itemize}
    \item{\textbf{Địa chỉ}} 01 Công xã Paris, Bến Nghé, Quận 1, Thành phố Hồ Chí Minh
    \item{\textbf{Giờ mở cửa:}} 5h30 - 17h
    \item{\textbf{Giá vé/ chi phí:}} Miễn phí
    \item{\textbf{Phân loại điểm du lịch} }
    \item{\textbf{Nguồn thông tin}}
    \item{\textbf{Thông tin về địa điểm:}}
\end{itemize}
Nhà thờ Đức Bà là một biểu tượng của TPHCM và một trong những công trình kiến trúc nổi tiếng nhất của thành phố. Với kiến trúc hùng vĩ và điểm nhấn là những cửa sổ nghệ thuật đẹp mắt, nhà thờ này là điểm đến thu hút rất nhiều du khách hàng năm. Bên trong, bạn có thể chiêm ngưỡng những bức tranh tường và tượng gỗ cầu kỳ, tạo nên một không gian tôn giáo trang nghiêm. Nhà thờ Đức Bà cũng là một di tích lịch sử quan trọng, đại diện cho sự gắn kết văn hóa và tôn giáo trong lịch sử phát triển của thành phố.
\begin{itemize}
    \item{\textbf{Các thông tin khác - Cách di chuyển}}
\end{itemize}

\section{Bảo tàng Thành phố Hồ Chí Minh}
\begin{itemize}
    \item{\textbf{Địa chỉ}} 65 Lý Tự Trọng, Bến Nghé, Quận 1, Thành phố Hồ Chí Minh
    \item{\textbf{Giờ mở cửa:}} Thứ ba, thứ tư, thứ năm, thứ bảy, chủ nhật: 8h - 12h, 14h - 16h30; Thứ hai và thứ sáu: 8h - 12h
    \item{\textbf{Giá vé/ chi phí:}} 40.000đ cho du khách quốc tế, Khách Việt Nam được miễn phí
    \item{\textbf{Phân loại điểm du lịch} }
    \item{\textbf{Nguồn thông tin}}
    \item{\textbf{Thông tin về địa điểm:}}
\end{itemize}
Bảo tàng Thành phố Hồ Chí Minh là một điểm đến thú vị để tìm hiểu về lịch sử và văn hóa của thành phố. Được xây dựng theo lối kiến trúc phục hưng, nơi đây không chỉ lưu giữ những tư liệu cùng hiện vật lịch sử của Thành phố Hồ Chí Minh mà còn là nơi trưng bày các cổ vật, nét văn hóa của ông cha ta thời xưa. Bên cạnh đó, với khuôn viên rộng lớn được bao phủ bởi rất nhiều cây xanh đem đến cảm giác trong lành, thư giãn tránh xa được những ồn ào của phố thị. Đây là một điểm dừng chân lý tưởng để tìm hiểu và khám phá lịch sử, văn hóa của con người tại Thành phố mang tên Bác.
\begin{itemize}
    \item{\textbf{Các thông tin khác - Cách di chuyển}}
\end{itemize}

\section{Dinh Độc Lập}
\begin{itemize}
    \item{\textbf{Địa chỉ}} Số 135 đường Nam Kỳ Khởi Nghĩa, thành phố Hồ Chí Minh
    \item{\textbf{Giờ mở cửa:}} 8h - 16h30
    \item{\textbf{Giá vé/ chi phí:}} 10.000đ - 65.000đ
    \item{\textbf{Phân loại điểm du lịch} }
    \item{\textbf{Nguồn thông tin}}
    \item{\textbf{Thông tin về địa điểm:}}
\end{itemize}
Dinh Độc Lập được khởi công xây dựng ngày 1/7/1962 theo đồ án thiết kế của kiến trúc sư Ngô Viết Thụ. Nơi đây lưu giữ những dấu mốc đáng tự hào của dân tộc trong cuộc chiến tranh bảo vệ đất nước vào thế kỷ trước. Đặc biệt là sự kiện kết thúc chiến tranh tại miền Nam, thống nhất đất nước vào ngày 30 tháng 4 năm 1975. Đến năm 1976, Dinh Độc Lập được gọi là Hội trường Thống Nhất. Ngày nay, khi đến tham quan Dinh Độc Lập, để tìm hiểu thêm nhiều chi tiết lịch sử thú vị, du khách có thể tham quan khu vực trưng bày thông tin và tư liệu về Dinh Norodom (1868) trước, rồi sau đó di chuyển qua khu vực tham quan phòng thông tin về toà nhà chính Dinh Độc Lập (1966) để hành trình trọn vẹn hơn. Bên trong dinh còn có cửa hàng quà lưu niệm, nhà hàng và quán cà phê để phục vụ du khách tham quan.
\begin{itemize}
    \item{\textbf{Các thông tin khác - Cách di chuyển}}
\end{itemize}

\section{Địa đạo Củ Chi}
\begin{itemize}
    \item{\textbf{Địa chỉ}} Phú Hiệp, Củ Chi, Thành phố Hồ Chí Minh
    \item{\textbf{Giờ mở cửa:}} 7h - 17h
    \item{\textbf{Giá vé/ chi phí:}} 35.000đ
    \item{\textbf{Phân loại điểm du lịch} }
    \item{\textbf{Nguồn thông tin}}
    \item{\textbf{Thông tin về địa điểm:}}
\end{itemize}
Địa đạo Củ Chi là một điểm du lịch nổi tiếng ở Việt Nam, nằm cách trung tâm thành phố Hồ Chí Minh khoảng 70km về hướng Tây Bắc. Ngày nay, địa đạo Củ Chi đã được bảo tồn và trở thành một điểm tham quan lịch sử hấp dẫn đặc biệt với các du khách nước ngoài.

Du khách đến địa đạo Củ Chi sẽ có cơ hội trải nghiệm những hoạt động độc đáo liên quan đến cuộc sống trong địa đạo. Các hoạt động bao gồm khám phá các hầm, tham gia vào các trò chơi quân đội, rèn luyện bắn súng, trải nghiệm ăn những món ăn trong thời chiến cùng với người dân Củ Chi.

\begin{itemize}
    \item{\textbf{Các thông tin khác - Cách di chuyển}}
\end{itemize}

\section{Bảo tàng Mỹ thuật}
\begin{itemize}
    \item{\textbf{Địa chỉ}} 97A Phó Đức Chính, Phường Nguyễn Thái Bình, Quận 1, Thành phố Hồ Chí Minh
    \item{\textbf{Giờ mở cửa:}} 8h - 17h
    \item{\textbf{Giá vé/ chi phí:}} 15.000đ - 30.000đ (Miễn phí: Cho trẻ em dưới 6 tuổi và người trên 60 tuổi)
    \item{\textbf{Phân loại điểm du lịch} }
    \item{\textbf{Nguồn thông tin}}
    \item{\textbf{Thông tin về địa điểm:}}
\end{itemize}
Bảo tàng Mỹ thuật là một viện bảo tàng nghệ thuật ấn tượng tại TP.HCM. Được thành lập từ năm 1975, bảo tàng này chứa đựng nhiều tác phẩm nghệ thuật đặc sắc của Việt Nam và thế giới. Du khách có thể tìm hiểu về lịch sử và phong cách nghệ thuật qua các triển lãm và bộ sưu tập đa dạng tại đây. Bảo tàng Mỹ thuật là điểm đến lý tưởng của rất nhiều du khách, đặc biệt là những bạn thích khám phá vẻ đẹp của các tác phẩm nghệ thuật.

\begin{itemize}
    \item{\textbf{Các thông tin khác - Cách di chuyển}}
\end{itemize}

\section{Hồ Bán Nguyệt}
\begin{itemize}
    \item{\textbf{Địa chỉ}} Khu đô thị mới Phú Mỹ Hưng, quận 7, thành phố Hồ Chí Minh
    \item{\textbf{Giờ mở cửa:}} Mở cửa cả ngày
    \item{\textbf{Giá vé/ chi phí:}} Miễn phí
    \item{\textbf{Phân loại điểm du lịch} }
    \item{\textbf{Nguồn thông tin}}
    \item{\textbf{Thông tin về địa điểm:}}
\end{itemize}
Nằm giữa trung tâm thành phố, Hồ Bán Nguyệt là một hồ nước lớn tại TP.HCM. Điểm đặc biệt của hồ này chính là không gian yên bình và thoáng đãng mà nó mang lại. Du khách có thể tận hưởng cảnh quan hồ nước tươi mát trong lành, ngắm nhìn những đám mây trôi qua trên bầu trời xanh biếc, và thể hiện sự thư giãn trong cuộc sống hối hả của thành phố. Ngoài ra, tại Hồ Bán Nguyệt còn có các hoạt động giải trí như chèo thuyền, câu cá và tản bộ quanh hồ.

\begin{itemize}
    \item{\textbf{Các thông tin khác - Cách di chuyển}}
\end{itemize}

\section{Chùa Vĩnh Nghiêm}
\begin{itemize}
    \item{\textbf{Địa chỉ}} 339 Nam Kỳ Khởi Nghĩa, Phường 7, Quận 3, Thành phố Hồ Chí Minh
    \item{\textbf{Giờ mở cửa:}} 7h - 21h
    \item{\textbf{Giá vé/ chi phí:}} Miễn phí
    \item{\textbf{Phân loại điểm du lịch} }
    \item{\textbf{Nguồn thông tin}}
    \item{\textbf{Thông tin về địa điểm:}}
\end{itemize}
Chùa Vĩnh Nghiêm là một công trình kiến trúc tôn giáo nổi tiếng tọa lạc tại TP.HCM. Được xây dựng vào những năm 1960, ngôi chùa này là điểm đến thú vị để tham quan và tìm hiểu về đạo Phật. Chùa Vĩnh Nghiêm nổi tiếng với kiến trúc lớn và đẹp, được xây dựng theo phong cách truyền thống của Việt Nam. Du khách có thể tham quan các tòa tháp, điện và những bức tượng Phật tại chùa này để hiểu thêm về lịch sử và đạo Phật.

\begin{itemize}
    \item{\textbf{Các thông tin khác - Cách di chuyển}}
\end{itemize}

\section{Bưu điện Thành Phố Hồ Chí Minh}
\begin{itemize}
    \item{\textbf{Địa chỉ}} 02 Công xã Paris, Bến Nghé, Quận 1, Thành phố Hồ Chí Minh
    \item{\textbf{Giờ mở cửa:}} 7h30 - 18h
    \item{\textbf{Giá vé/ chi phí:}} Miễn phí
    \item{\textbf{Phân loại điểm du lịch} }
    \item{\textbf{Nguồn thông tin}}
    \item{\textbf{Thông tin về địa điểm:}}
\end{itemize}
Bưu điện TPHCM, còn được gọi là Bưu điện Trung tâm Sài Gòn, là một công trình kiến trúc nổi tiếng và biểu tượng của thành phố. Kiến trúc kết hợp giữa phong cách Châu Âu với các yếu tố kiến trúc Đông Dương, tạo nên một diện mạo độc đáo và lôi cuốn.

Bên trong, bưu điện TPHCM được thiết kế sang trọng và đẹp mắt. Hành lang dẫn vào bưu điện trang trí bằng những bức tranh và tượng điêu khắc tuyệt đẹp. Ngoài ra, du khách có thể ghé thăm phòng Bưu chính Việt Nam để tìm hiểu thêm về lịch sử bưu chính và viễn thông của Việt Nam.

\begin{itemize}
    \item{\textbf{Các thông tin khác - Cách di chuyển}}
\end{itemize}

\section{Khu du lịch Văn Thánh}
\begin{itemize}
    \item{\textbf{Địa chỉ}: 48/10 Đường Điện Biên Phủ, Phường 22, Bình Thạnh, Thành phố Hồ Chí Minh}
    \item{\textbf{Giờ mở cửa:} 10h - 22h}
    \item{\textbf{Giá vé/ chi phí:} Miễn phí vào cổng (Du khách chỉ chi trả cho các dịch vụ khi sử dụng)}
    \item{\textbf{Phân loại điểm du lịch} }
    \item{\textbf{Nguồn thông tin}}
    \item{\textbf{Thông tin về địa điểm:}}
\end{itemize}
Khu du lịch Văn Thánh là một khu vườn rộng lớn nằm ở quận Bình Thạnh, TP.HCM. Đây là một địa điểm lý tưởng để dạo chơi và tổ chức picnic cuối tuần. Khu vườn có không gian xanh mát, được trang trí bằng các loại cây cảnh và hồ nước. Du khách có thể đi dạo trong khu vườn, tham gia các hoạt động thể thao như cầu lông, bóng đá, và thưởng thức các món ăn vặt tại các gian hàng đường phố xung quanh.
\begin{itemize}
    \item{\textbf{Các thông tin khác - Cách di chuyển}}
\end{itemize}

\section{Nhà thờ Tân Định}
\begin{itemize}
    \item{\textbf{Địa chỉ}: 289 Hai Bà Trưng, Phường 8, Quận 3, Thành phố Hồ Chí Minh}
    \item{\textbf{Giờ mở cửa:} Thứ Hai đến thứ Sáu: 8h - 11h30, 14h - 17h}
    \item{\textbf{Giá vé/ chi phí:} Miễn phí}
    \item{\textbf{Phân loại điểm du lịch} }
    \item{\textbf{Nguồn thông tin}}
    \item{\textbf{Thông tin về địa điểm:}}
\end{itemize}
Nhà thờ Tân Định là một công trình kiến trúc tôn giáo nổi tiếng tọa lạc ở TP.HCM. Được xây dựng vào năm 1876, nhà thờ này nổi tiếng với màu hồng chủ đạo độc đáo, tạo nên vẻ đẹp đặc biệt. Du khách có thể tham quan bên trong nhà thờ để chiêm ngưỡng kiến trúc lịch sử của nó. Đây cũng là nơi tổ chức các nghi lễ tôn giáo quan trọng hàng năm, thu hút rất đông khách du lịch ghé thăm và chụp ảnh miễn phí.
\begin{itemize}
    \item{\textbf{Các thông tin khác - Cách di chuyển}}
\end{itemize}

\section{Khu du lịch Suối Tiên}
\begin{itemize}
    \item{\textbf{Địa chỉ}: 120 Đường Võ Nguyên Giáp, Thành Phố, Thủ Đức, Thành phố Hồ Chí Minh}
    \item{\textbf{Giờ mở cửa:} 7h30 - 17h}
    \item{\textbf{Giá vé/ chi phí:} 80.000đ - 150.000đ}
    \item{\textbf{Phân loại điểm du lịch} }
    \item{\textbf{Nguồn thông tin}}
    \item{\textbf{Thông tin về địa điểm:}}
\end{itemize}
Khu du lịch Suối Tiên là một công viên giải trí lớn tọa lạc ở quận 9, TP.HCM. Nơi đây có các trò chơi thú vị và tiểu cảnh đẹp mắt, thu hút rất nhiều du khách mỗi năm. Các hoạt động tại khu du lịch Suối Tiên bao gồm đi thuyền, trượt nước, đi dạo ngắm nhìn cảnh quan cũng như tận mắt thấy được nhiều động vật đáng yêu như hươu, voi, ngựa vằn, hà mã..., và tham gia các trò chơi mạo hiểm. Du khách có thể tận hưởng những giây phút vui vẻ, thư giãn cùng gia đình và bạn bè tại đây.
\begin{itemize}
    \item{\textbf{Các thông tin khác - Cách di chuyển}}
\end{itemize}

\section{Khu phố Nhật Bản (Little Tokyo)}
\begin{itemize}
    \item{\textbf{Địa chỉ}: hẻm số 15B đường Lê Thánh Tôn, Quận 1, Thành phố Hồ Chí Minh}
    \item{\textbf{Giờ mở cửa:} Mở cửa cả ngày}
    \item{\textbf{Giá vé/ chi phí:} Miễn phí (Du khách chỉ chi trả khi sử dụng dịch vụ)}
    \item{\textbf{Phân loại điểm du lịch} }
    \item{\textbf{Nguồn thông tin}}
    \item{\textbf{Thông tin về địa điểm:}}
\end{itemize}
Khu phố Nhật Bản, hay còn được gọi là Little Tokyo, là một khu vực tập trung nhiều nhà hàng, quán cà phê và cửa hàng Nhật Bản tại TP.HCM. Đây là điểm đến mang đến không gian văn hóa Nhật đặc biệt, nơi du khách có thể thưởng thức các món ăn Nhật truyền thống, mua sắm các sản phẩm và trang phục Nhật Bản, và trải nghiệm văn hoá độc đáo của Nhật Bản ngay tại Việt Nam.
\begin{itemize}
    \item{\textbf{Các thông tin khác - Cách di chuyển}}
\end{itemize}

\section{Công viên Cầu Ánh Sao (Starlight Bridge Park)}
\begin{itemize}
    \item{\textbf{Địa chỉ}: khu đô thị mới Phú Mỹ Hưng, quận 7, Thành phố Hồ Chí Minh}
    \item{\textbf{Giờ mở cửa:} Mở cửa cả ngày (Du khách nên đến vào 17h30 là thời điểm đẹp nhất)}
    \item{\textbf{Giá vé/ chi phí:} Miễn phí}
    \item{\textbf{Phân loại điểm du lịch} }
    \item{\textbf{Nguồn thông tin}}
    \item{\textbf{Thông tin về địa điểm:}}
\end{itemize}
Công viên Cầu Ánh Sao là một công viên với cầu ánh sáng đẹp mắt tọa lạc ở quận 7, TP.HCM. Được thiết kế theo phong cách hiện đại và lãng mạn, công viên Cầu Ánh Sao tạo ra không gian thư giãn và lãng mạn cho người dân và du khách.
\begin{itemize}
    \item{\textbf{Các thông tin khác - Cách di chuyển}}
\end{itemize}

\section{Nhà thờ Đức Bà biểu tượng}
\begin{itemize}
    \item{\textbf{Địa chỉ}} Số 1 Công xã Paris, phường Bến Nghé, Quận 1, TP. HCM
    \item{\textbf{Giờ mở cửa:}} 8h30 – 10h30 và 14h30 – 15h30 (Từ thứ 2 đến thứ 6)
    \item{\textbf{Giá vé/ chi phí:}} Miễn phí
    \item{\textbf{Phân loại điểm du lịch} }
    \item{\textbf{Nguồn thông tin}}
    \item{\textbf{Thông tin về địa điểm:}} Có thể nói nhà thờ Đức Bà Sài Gòn là một trong những địa điểm du lịch nổi bật nhất nơi đây, còn được biết đến với cái tên khác là Nhà thờ chính tòa Đức Bà Sài Gòn hay Vương cung thánh đường chính tòa Đức Mẹ Vô nhiễm Nguyên tội. Bên trong thánh đường được thiết kế thành 1 lòng chính, 2 lòng phụ cùng 2 dãy nhà nguyện. Phía trước thánh đường có công viên với 4 đường giao nhau tạo thành hình thánh giá.
\end{itemize}

\section{Dinh Độc Lập nơi ghi dấu ấn vàng son ngày thống nhất đất nước}
\begin{itemize}
    \item{\textbf{Địa chỉ}} 135 Nam Kỳ Khởi Nghĩa, Quận 1, TP. HCM
    \item{\textbf{Giờ mở cửa:}} 7h30 – 11h00 và 13h00 – 16h00 tất cả các ngày trong tuần.
    \item{\textbf{Giá vé/ chi phí:}} 40.000đ/người lớn. Sinh viên 20.000đ/vé. Học sinh (từ 6 tuổi đến dưới 16 tuổi): 10.000đ/vé
    \item{\textbf{Phân loại điểm du lịch} }
    \item{\textbf{Nguồn thông tin}}
    \item{\textbf{Thông tin về địa điểm:}} Biểu tượng của du lịch Sài Gòn – Dinh Độc Lập (hay dinh Thống Nhất) là nơi Toàn quyền Đông Dương và tổng thống Ngô Đình Diệm làm việc trước sự kiện 30/4/1975 lịch sử. Công trình kiến trúc ấn tượng này ghi dấu ấn đậm nét trong giai đoạn những năm 1960.
\end{itemize}

\section{Chiêm ngưỡng các tác phẩm nghệ thuật đặc sắc tại Bảo tàng Mỹ thuật TP.HCM}
\begin{itemize}
    \item{\textbf{Địa chỉ}} 97A Phó Đức Chính, Quận 1, TP. HCM
    \item{\textbf{Giờ mở cửa:}} 8h – 17h các ngày trong tuần
    \item{\textbf{Giá vé/ chi phí:}} 30.000đ/người lớn. Học sinh sinh viên: 15.000đ. Trẻ em dưới 6 tuổi và người trên 60 tuổi: Miễn phí.
    \item{\textbf{Phân loại điểm du lịch} }
    \item{\textbf{Nguồn thông tin}}
    \item{\textbf{Thông tin về địa điểm:}} Một địa điểm du lịch Sài Gòn mang đậm giá trị nghệ thuật và lịch sử, trước đây là căn nhà của gia đình ông Hứa Bổn Hòa, một trong tứ đại gia Sài Gòn cuối thế kỷ 19. Bảo tàng Mỹ thuật TP. HCM mang kiến trúc Pháp cổ cuối thế kỷ 19 – đầu thế kỷ 20, mang nét hài hòa giữa Đông và Tây nhờ mái ngói, cột ốp gốm và hoa văn đậm chất dân gian của những viên gạch dưới sàn nhà… Bảo tàng Mỹ thuật còn là điểm đến chụp hình yêu thích của các bạn trẻ
\end{itemize}

\begin{itemize}
    \item{\textbf{Các thông tin khác - Cách di chuyển}}
\end{itemize}

\section{Chợ Bến Thành - Biểu tượng của văn hóa Sài Gòn}
\begin{itemize}
    \item{\textbf{Địa chỉ}} Phường Bến Thành, Quận 1, TP. HCM
    \item{\textbf{Giờ mở cửa:}} 7h00 – 19h00
    \item{\textbf{Giá vé/ chi phí:}} Miễn phí
    \item{\textbf{Phân loại điểm du lịch} }
    \item{\textbf{Nguồn thông tin}}
    \item{\textbf{Thông tin về địa điểm:}} Chợ Bến Thành hình thành giữa thế kỷ 19. Trải qua thăng trầm của lịch sử, khu chợ này từ lâu đã trở thành biểu tượng cho văn hóa mua bán sôi động của Sài Gòn nói riêng và Việt Nam nói chung. Cùng với sự phát triển kinh tế, du lịch của Thành phố Hồ Chí Minh, khu chợ này cũng ngày càng nổi tiếng và đặc biệt thu hút đông đảo khách du lịch nước ngoài ghé đến tham quan.
\end{itemize}

\begin{itemize}
    \item{\textbf{Các thông tin khác - Cách di chuyển}}
\end{itemize}

\section{Bến Nhà Rồng - Chứng tích rạng ngời của lịch sử dân tộc}
\begin{itemize}
    \item{\textbf{Địa chỉ}} Số 01 Nguyễn Tất Thành, Phường 12, Quận 4, TP. HCM
    \item{\textbf{Giờ mở cửa:}} 7h30 - 11h30 và 13h30 - 17h30 (không mở cửa vào Thứ Hai)
    \item{\textbf{Giá vé/ chi phí:}} 20.000 VNĐ/người (miễn phí cho học sinh, sinh viên), 2.000 VNĐ giữ xe
    \item{\textbf{Phân loại điểm du lịch} }
    \item{\textbf{Nguồn thông tin}}
    \item{\textbf{Thông tin về địa điểm:}} Bến Nhà Rồng là một di tích lịch sử nổi tiếng ở trung tâm Thành phố Hồ Chí Minh. Đây là điểm tham quan gắn liền với câu chuyện đi tìm đường cứu nước của Chủ tịch Hồ Chí Minh. Sau khi chiến tranh miền Nam kết thúc, Bến Nhà Rồng đã được tu sửa và cải tạo lại công trình. Hiện nay, Bến Nhà Rồng có bốn khu vực chính là Bảo tàng Hồ Chí Minh, tượng đài Bác Hồ, khuôn viên bến cảng và đài phun nước. Các tín đồ du lịch vẫn thường hay gọi tòa nhà bảo tàng là Nhà Rồng và bến cảng ở ngay bên cạnh là Bến Nhà Rồng.
\end{itemize}

\begin{itemize}
    \item{\textbf{Các thông tin khác - Cách di chuyển}}
\end{itemize}

\section{Địa đạo Củ Chi}
\begin{itemize}
    \item{\textbf{Địa chỉ}} Tỉnh lộ 15, ấp Phú Hiệp, xã Phú Mỹ Hưng, huyện Củ Chi, TPHCM
    \item{\textbf{Giờ mở cửa:}} 7h00 – 17h00
    \item{\textbf{Giá vé/ chi phí:}} 20.000đ/người Việt, 110.000đ/khách nước ngoài
    \item{\textbf{Phân loại điểm du lịch} }
    \item{\textbf{Nguồn thông tin}}
    \item{\textbf{Thông tin về địa điểm:}} Khu di tích Địa đạo Củ Chi cách trung tâm thành phố khá xa nhưng rất được khách du lịch nước ngoài yêu thích. Trước đây địa đạo từng là hệ thống phòng thủ, căn cứ kháng chiến của quân dân miền Nam. Lán che xuống hầm địa đạo, nơi vẫn còn được giữ nguyên trạng cho đến ngày nay
\end{itemize}

\begin{itemize}
    \item{\textbf{Các thông tin khác - Cách di chuyển}}
\end{itemize}

\section{Dinh Gia Long gắn liền với những biến động lịch sử quan trọng}
\begin{itemize}
    \item{\textbf{Địa chỉ}} 65 Đường Lý Tự Trọng, Phường Bến Nghé, Quận 1, TP. HCM
    \item{\textbf{Giờ mở cửa:}} 08h00 - 17h00
    \item{\textbf{Giá vé/ chi phí:}} 30.000 VNĐ/lượt (mỗi lượt tham quan 2 tiếng)
    \item{\textbf{Phân loại điểm du lịch} }
    \item{\textbf{Nguồn thông tin}}
    \item{\textbf{Thông tin về địa điểm:}} Nhắc đến dinh thự lịch sử ở Thành phố Hồ Chí Minh, nhiều người sẽ chỉ biết đến Dinh Độc Lập, còn Dinh Gia Long thì chưa nghe đến bao giờ. Thực chất Dinh Gia Long hiện nay đã trở thành Bảo tàng Thành phố Hồ Chí Minh, nơi lưu giữ rất nhiều hiện vật quan trọng phản ánh lại một giai đoạn lịch sử đầy biến động của Sài Gòn. Dinh Gia Long nằm cách Dinh Độc Lập chỉ một con đường, giữa trung tâm quận 1 sầm uất. Vì vậy, với những bạn đến du lịch Sài Gòn thì chắc chắn không thể nào bỏ lỡ 2 địa điểm tham quan này. Dù đã trở thành Bảo tàng Thành phố Hồ Chí Minh nhưng Dinh Gia Long vẫn được nhắc đến với rất nhiều câu chuyện lịch sử và cả giai thoại thú vị. Dinh Gia Long nay đã trở thành Bảo tàng Thành phố Hồ Chí Minh
\end{itemize}

\begin{itemize}
    \item{\textbf{Các thông tin khác - Cách di chuyển}}
\end{itemize}

\section{Tòa nhà cao nhất Việt Nam - Landmark 81}
\begin{itemize}
    \item{\textbf{Địa chỉ}} 720A Điện Biên Phủ, quận Bình Thạnh, TP. HCM
    \item{\textbf{Giờ mở cửa:}}
    \item{\textbf{Giá vé/ chi phí:}}
    \item{\textbf{Phân loại điểm du lịch} }
    \item{\textbf{Nguồn thông tin}}
    \item{\textbf{Thông tin về địa điểm:}}
\end{itemize}
Landmark 81 là công trình kiến trúc bề thế tại trung tâm thành phố với chiều cao lên đến 470m và là tòa nhà cao nhất nước ta tính đến nay. Tại địa điểm du lịch Sài Gòn này bạn sẽ được thỏa chí vui chơi, tham gia vào nhiều dịch vụ tiện ích và mua sắm thả ga. Những địa điểm được yêu thích nhất ở đây là: rạp CGV, sân băng Vincom Ice Rink… Landmark 81 có chiều cao lên đến 470m
\begin{itemize}
    \item{\textbf{Các thông tin khác - Cách di chuyển}}
\end{itemize}

\section{Tòa nhà Bitexco - Minh chứng cho sự thịnh vượng và phát triển của TP. HCM}
\begin{itemize}
    \item{\textbf{Địa chỉ}} 2 Hải Triều, phường Bến Nghé, quận 1, TP. HCM
    \item{\textbf{Giờ mở cửa:}}
    \item{\textbf{Giá vé/ chi phí:}}
    \item{\textbf{Phân loại điểm du lịch} }
    \item{\textbf{Nguồn thông tin}}
    \item{\textbf{Thông tin về địa điểm:}}
\end{itemize}
Bitexco Financial Tower (hay còn được gọi tắt là Bitexco Tower) là một tòa nhà có độ cao thứ hai tại Thành phố Hồ Chí Minh (sau Landmark 81). Tòa nhà Bitexco có 68 tầng nổi và 3 tầng hầm và được hoàn thiện vào giai đoạn cuối năm 2010. Vì số lượng tầng là 68, thế nên, Bitexco còn có một trung tâm thương mại với tên gọi là Icon68.
\begin{itemize}
    \item{\textbf{Các thông tin khác - Cách di chuyển}}
\end{itemize}

\section{Check-in, uống cà phê tại chung cư 42 Nguyễn Huệ}
\begin{itemize}
    \item{\textbf{Địa chỉ}} số 42 đường Nguyễn Huệ, phường Bến Nghé, Quận 1, TP. HCM
    \item{\textbf{Giờ mở cửa:}}
    \item{\textbf{Giá vé/ chi phí:}}
    \item{\textbf{Phân loại điểm du lịch} }
    \item{\textbf{Nguồn thông tin}}
    \item{\textbf{Thông tin về địa điểm:}}
\end{itemize}
Chung cư 42 Nguyễn Huệ là điểm check-in sống ảo và tham quan nóng hổi dành cho bất cứ ai đến với thành phố diễm lệ này. Bên ngoài, tòa nhà này trông như một khối rubik 9 tầng với rất nhiều ô đa sắc màu, nhờ thế, dù ban ngày hay ban đêm, nơi đây vẫn luôn nổi bật theo cách riêng của mình. Nơi đây cũng tập trung rất nhiều quán cafe, nhà hàng và cửa hàng thời trang. Vì vậy, khi đến đây, bạn vừa có thể thả hồn mua sắm hoặc vừa thư giãn với một cốc cà phê sữa béo thơm vừa nhìn ngắm toàn cảnh Sài Gòn từ trên cao. Ngoài ra đến với chung cư 42 Nguyễn Huệ bạn còn có thể chụp cháy máy với những tấm ảnh xịn xò nét cực căng.
\begin{itemize}
    \item{\textbf{Các thông tin khác - Cách di chuyển}}
\end{itemize}

\section{Check-in cháy máy ở Phố đi bộ Nguyễn Huệ}
\begin{itemize}
    \item{\textbf{Địa chỉ}} Đường Nguyễn Huệ, Quận 1, TP. HCM
    \item{\textbf{Giờ mở cửa:}} Cả ngày
    \item{\textbf{Giá vé/ chi phí:}} Miễn phí
    \item{\textbf{Phân loại điểm du lịch} }
    \item{\textbf{Nguồn thông tin}}
    \item{\textbf{Thông tin về địa điểm:}}
\end{itemize}
Nếu Hà Nội có Hồ Gươm thì du lịch Sài Gòn cũng có phố đi bộ Nguyễn Huệ là điểm đến được yêu thích vào mỗi cuối tuần. Từ khi đoạn đường dài 670m và rộng 64m này được đưa vào khai thác từ năm 2005 thì nó đã trở thành điểm đến hàng đầu của người dân và cả giới trẻ địa phương. Bên cạnh đài phun nước lớn, hệ thống cây xanh, chúng ta còn có cơ hội được thưởng thức ẩm thực Sài Gòn đặc sắc trên phố đi bộ.
\begin{itemize}
    \item{\textbf{Các thông tin khác - Cách di chuyển}}
\end{itemize}

\section{Xập xình xuyên đêm tại phố đi bộ Bùi Viện}
\begin{itemize}
    \item{\textbf{Địa chỉ}} Đường Bùi Viện, Phường Phạm Ngũ Lão, Quận 1, TP. HCM
    \item{\textbf{Giờ mở cửa:}}
    \item{\textbf{Giá vé/ chi phí:}}
    \item{\textbf{Phân loại điểm du lịch} }
    \item{\textbf{Nguồn thông tin}}
    \item{\textbf{Thông tin về địa điểm:}}
\end{itemize}
Từ lâu, phố đi bộ Bùi Viện đã trở thành một tụ điểm vui chơi nhộn nhịp và nổi tiếng trong cộng đồng các bạn trẻ ở quận 1. Dưới những ánh đèn neon đủ màu sắc cùng âm nhạc rộn ràng vang vọng, đến với phố Bùi Viện những bạn trẻ đam mê không khí nhộn nhịp sôi động như tìm được một phần của chính mình.
\begin{itemize}
    \item{\textbf{Các thông tin khác - Cách di chuyển}}
\end{itemize}

\section{Khám phá các loài động thực vật tại Thảo Cầm Viên}
\begin{itemize}
    \item{\textbf{Địa chỉ}} 2 Nguyễn Bỉnh Khiêm, phường Bến Nghé, Quận 1, TP. HCM
    \item{\textbf{Giờ mở cửa:}} 7h00 – 17h30
    \item{\textbf{Giá vé/ chi phí:}} 60.000đ/người cao trên 1.3m, 40.000đ/người cao từ 1 – 1.3m, trẻ em dưới 1m miễn phí
    \item{\textbf{Phân loại điểm du lịch} }
    \item{\textbf{Nguồn thông tin}}
    \item{\textbf{Thông tin về địa điểm:}}
\end{itemize}
Với nhiều khách du lịch Sài Gòn đi cùng gia đình, việc trải nghiệm một ngày ở Sở Thú sẽ mang đến nhiều kỷ niệm đáng nhớ. Thảo Cầm Viên không những là nơi bảo tồn 1000 cá thể động vật mà còn sở hữu hơn 2000 cây hoa, cây cảnh độc đáo. Tại đó chúng ta sẽ được tham gia vào nhiều trò chơi lý thú như: câu cá, xe lửa, ngựa quay… Các bé có thể tự tay cho một số loài vật ăn tại Thảo Cầm Viên.
\begin{itemize}
    \item{\textbf{Các thông tin khác - Cách di chuyển}}
\end{itemize}

\section{Loạt trải nghiệm hấp dẫn tại khu du lịch Suối Tiên}
\begin{itemize}
    \item{\textbf{Địa chỉ}} 120 Xa lộ Hà Nội, phường Tân Phú, Quận 9, TP. HCM
    \item{\textbf{Giờ mở cửa:}} 6h30 – 22h00
    \item{\textbf{Giá vé/ chi phí:}} 120.000đ/người lớn, 60.000đ/trẻ em từ 1 – 1.4m, trẻ em dưới 1m miễn phí
    \item{\textbf{Phân loại điểm du lịch} }
    \item{\textbf{Nguồn thông tin}}
    \item{\textbf{Thông tin về địa điểm:}}
\end{itemize}
Công viên văn hóa Suối Tiên gây ấn tượng mạnh với du khách gần xa khi tích hợp đủ loại hình vui chơi hiện đại, độc đáo với phong cách tri ân văn hóa ấn tượng. Chẳng hạn như các công trình được xây theo thần thoại, truyền thuyết Việt Nam như: Sơn Tinh – Thủy Tinh, Vua Hùng, Lạc Long Quân – Âu Cơ… Những trò chơi được yêu thích nhất tại đây có thể kể đến như: rạp phim 3D/4D, phi thuyền đại chiến các vì sao, đường đua xe siêu tốc… Biển Tiên Đồng tại khu du lịch Suối Tiên.
\begin{itemize}
    \item{\textbf{Các thông tin khác - Cách di chuyển}}
\end{itemize}

\section{Giải nhiệt ở Công viên nước Đầm Sen}
\begin{itemize}
    \item{\textbf{Địa chỉ}} Số 3 đường Hòa Bình, Phường 3, Quận 11, TP. HCM
    \item{\textbf{Giờ mở cửa:}}
    \item{\textbf{Giá vé/ chi phí:}}
    \item{\textbf{Phân loại điểm du lịch} }
    \item{\textbf{Nguồn thông tin}}
    \item{\textbf{Thông tin về địa điểm:}}
\end{itemize}
Công viên nước Đầm Sen là một trong những điểm đến hàng đầu ở Sài Thành, nơi bạn được thả mình vào các trò chơi giải trí đa dạng và thú vị, phục vụ mọi sở thích và độ tuổi, mang đến cho bạn những trải nghiệm tuyệt vời. Công viên nước Đầm Sen là nơi giải nhiệt hàng đầu ở Sài Thành.
\begin{itemize}
    \item{\textbf{Các thông tin khác - Cách di chuyển}}
\end{itemize}

\section{"Phá đảo" chợ ẩm thực Hồ Thị Kỷ}
\begin{itemize}
    \item{\textbf{Địa chỉ}} Đường Hồ Thị Kỷ, Phường 1, Quận 10, TP. HCM
    \item{\textbf{Giờ mở cửa:}} Bắt đầu từ 15h00 đến 23h00 (Những ngày lễ, Tết chợ hoạt động xuyên đêm)
    \item{\textbf{Giá vé/ chi phí:}}
    \item{\textbf{Phân loại điểm du lịch} }
    \item{\textbf{Nguồn thông tin}}
    \item{\textbf{Thông tin về địa điểm:}}
\end{itemize}
Chợ Hồ Thị Kỷ được biết đến là một khu chợ hoa đầu mối lớn nhất Tp.HCM, nhưng lại nổi tiếng với danh xưng “thiên đường ẩm thực” sầm uất của giới trẻ Sài Thành. Đến với chợ ẩm thực Hồ Thị Kỷ, các bạn sẽ phải “phát thèm” với những món ăn vặt ngon nức mũi. Chợ ẩm thực Hồ Thị Kỷ sẽ làm bạn sẽ phải “phát thèm”.
\begin{itemize}
    \item{\textbf{Các thông tin khác - Cách di chuyển}}
\end{itemize}

\section{Khu du lịch Văn Thánh}
\begin{itemize}
    \item{\textbf{Địa chỉ}} 48/10 Điện Biên Phủ, phường 22, quận Bình Thạnh, TP. HCM
    \item{\textbf{Giờ mở cửa:}} tất cả các ngày trong tuần, từ 9h đến 17h
    \item{\textbf{Giá vé/ chi phí:}}
    \item{\textbf{Phân loại điểm du lịch} }
    \item{\textbf{Nguồn thông tin}}
    \item{\textbf{Thông tin về địa điểm:}}
\end{itemize}

Nằm ngay trên trục đường Điện Biên Phủ, Khu du lịch Văn Thánh là điểm tham quan, vui chơi nổi tiếng của Sài thành, và chỉ cách khu vực trung tâm Quận 1 tầm chừng 2km. Dù nằm ngay giữa lòng phố thị, tuy nhiên, dường như Khu du lịch Văn Thánh là thế giới khác tách biệt khỏi nhịp sống náo nhiệt ngoài kia. Đến đây, bạn sẽ ngạc nhiên trước bầu không khí yên bình, không xô bồ, náo nhiệt, khói bụi. Lấy cảm hứng từ nông thôn xưa, mọi góc nơi đây đều gợi lại sự thân quen của những ngôi làng gắn liền với tuổi thơ bao người. Khu du lịch Văn Thánh là thế giới khác tách biệt khỏi nhịp sống náo nhiệt.

\begin{itemize}
    \item{\textbf{Các thông tin khác - Cách di chuyển}}
\end{itemize}

\section{Khu du lịch sinh thái Bình Quới}
\begin{itemize}
    \item{\textbf{Địa chỉ}} 1147 Bình Quới, Phường 28, quận Bình Thạnh, TP. HCM
    \item{\textbf{Giờ mở cửa:}} 9h00 – 17h00
    \item{\textbf{Giá vé/ chi phí:}} 130.000đ/người lớn, 90.000đ/trẻ em dưới 1m
    \item{\textbf{Phân loại điểm du lịch} }
    \item{\textbf{Nguồn thông tin}}
    \item{\textbf{Thông tin về địa điểm:}}
\end{itemize}

Khu du lịch Bình Quới cách trung tâm thành phố khoảng 8km, thích hợp cho những chuyến dã ngoại cùng gia đình. Đến đây bạn sẽ được tận hưởng không gian làng quê Nam Bộ cùng những hình ảnh mộc mạc quen thuộc như lu nước, cầu khỉ, xuồng ba lá… Nhiều trò chơi cũng được tái hiện đặc sắc như chèo thuyền, câu cá… Không gian xanh đậm chất Nam Bộ tại khu du lịch sinh thái Bình Quới.

\begin{itemize}
    \item{\textbf{Các thông tin khác - Cách di chuyển}}
\end{itemize}

\section{Rừng Sác Cần Giờ}
\begin{itemize}
    \item{\textbf{Địa chỉ}} Đường Rừng Sác, xã Long Hòa, huyện Cần Giờ, TP. HCM
    \item{\textbf{Giờ mở cửa:}} 07h30 - 17h00
    \item{\textbf{Giá vé/ chi phí:}}
    \item{\textbf{Phân loại điểm du lịch} }
    \item{\textbf{Nguồn thông tin}}
    \item{\textbf{Thông tin về địa điểm:}}
\end{itemize}

Chỉ cách nội đô thành phố khoảng chừng 50km, Rừng Sác Cần Giờ là điểm đến xanh dành cho những tâm hồn đam mê khám phá thiên nhiên, hoặc muốn tìm hiểu về trang sử vàng dân tộc. Được UNESCO vinh danh là Khu dự trữ sinh quyển thế giới đầu tiên tại Việt Nam, Rừng Sác Cần Giờ như lá phổi xanh giữa lòng phố thị, đồng thời là ‘nhà’ của rất nhiều loài động, thực vật quý, kể cả những cái tên nằm trong Sách Đỏ, như khỉ đuôi dài, cá sấu hoa, trăn đất, kỳ đà, tắc kè cùng nhiều loại chim, cò khác.

\begin{itemize}
    \item{\textbf{Các thông tin khác - Cách di chuyển}}
\end{itemize}

\section{Khu du lịch sinh thái Dần Xây}
\begin{itemize}
    \item{\textbf{Địa chỉ}} 1541 Đường Rừng Sác, Xã An Thới Đông, Huyện Cần Giờ, Thành phố Hồ Chí Minh
    \item{\textbf{Giờ mở cửa:}} 07h30 - 17h00
    \item{\textbf{Giá vé/ chi phí:}}
    \item{\textbf{Phân loại điểm du lịch} }
    \item{\textbf{Nguồn thông tin}}
    \item{\textbf{Thông tin về địa điểm:}}
\end{itemize}

Khu du lịch sinh thái Dần Xây sẵn sàng chào đón bạn với rất nhiều trải nghiệm độc đáo như nghỉ dưỡng ở chòi trên sông, cắm trại trong rừng ngập mặn, tham gia trồng rừng phòng hộ, trải nghiệm sống cùng người dân địa phương… Thật không ngoa khi nói khu du lịch Dần Xây như một ”ốc đảo xanh” tuyệt đẹp của thành phố mang tên Bác.

\begin{itemize}
    \item{\textbf{Các thông tin khác - Cách di chuyển}}
\end{itemize}

\section{Khu du lịch sinh thái Song Long}
\begin{itemize}
    \item{\textbf{Địa chỉ}} Số 10 đường D1, phường Long Phước, TP. Thủ Đức, TP. HCM
    \item{\textbf{Giờ mở cửa:}}
    \item{\textbf{Giá vé/ chi phí:}}
    \item{\textbf{Phân loại điểm du lịch} }
    \item{\textbf{Nguồn thông tin}}
    \item{\textbf{Thông tin về địa điểm:}}
\end{itemize}

Khu du lịch Song Long có vị trí đắc địa ngay bên dòng sông Đồng Nai và được bao phủ bởi những hàng cây xanh mát. Nơi đây vẫn có diện tích khá lớn, được thiết kế như “miền Tây thu nhỏ” với những con rạch cùng hàng dừa tỏa bóng mát.

\begin{itemize}
    \item{\textbf{Các thông tin khác - Cách di chuyển}}
\end{itemize}

\section{Đi buýt đường sông: View nhìn thành phố từ sông Sài Gòn}
\begin{itemize}
    \item{\textbf{Địa chỉ}} Bến Bạch Đằng, 2 Tôn Đức Thắng, phường Bến Nghé, quận 1
    \item{\textbf{Giờ mở cửa:}} 8 giờ 30 – 21 giờ
    \item{\textbf{Giá vé/ chi phí:}} 15.000 đồng/vé
    \item{\textbf{Phân loại điểm du lịch} }
    \item{\textbf{Nguồn thông tin}}
    \item{\textbf{Thông tin về địa điểm:}}
\end{itemize}

Địa điểm du lịch ở TPHCM được nhiều người trải nghiệm nhất chính là đi xe buýt đường sông. Tuyến buýt sông ở TPHCM mang đến cho bạn một góc nhìn mới mẻ về Sài Gòn. Bạn sẽ được xe buýt chở đến các điểm tham quan nổi tiếng như: bến Nhà Rồng, cảng Bến Nghé và các địa điểm nổi tiếng khác như: cầu Phú Mỹ, mũi Đèn Đỏ, bến phà Bình Khánh.

\begin{itemize}
    \item{\textbf{Các thông tin khác - Cách di chuyển}}
\end{itemize}

\section{Thưởng thức bữa tối trên du thuyền}
\begin{itemize}
    \item{\textbf{Địa chỉ}} 5 Nguyễn Tất Thành, quận 4, TPHCM hoặc 10B Tôn đức thắng, quận 1
    \item{\textbf{Giờ mở cửa:}} hàng đêm từ 18 giờ 30 – 21 giờ 30
    \item{\textbf{Giá vé/ chi phí:}} Từ 350.000 đồng (tùy vào thực đơn)
    \item{\textbf{Phân loại điểm du lịch} }
    \item{\textbf{Nguồn thông tin}}
    \item{\textbf{Thông tin về địa điểm:}}
\end{itemize}

Du thuyền sẽ đưa bạn đi ngắm cảnh Sài Gòn về đêm. Các địa điểm như bến Nhà Rồng, cảng Bến Nghé, cầu Sài Gòn về đêm sẽ đẹp hơn khi bạn đi xe buýt vào ban ngày vì chúng được trang trí thêm nhiều ánh đèn điện. Đặc biệt là bạn sẽ được ngắm toàn cảnh tòa nhà Landmark 81 lung linh, hoành tráng khi về đêm. Bữa tối trên du thuyền cũng được phục vụ theo cả phong cách Á và Âu hoặc theo combo tùy vào lựa chọn của bạn. Bữa tối sẽ có kèm theo các chương trình nghệ thuật, biểu diễn ca múa nhạc và ảo thuật để tăng thêm sự thú vị.

\begin{itemize}
    \item{\textbf{Các thông tin khác - Cách di chuyển}}
\end{itemize}

\section{Khám phá toàn cảnh thành phố từ xe buýt hai tầng}
\begin{itemize}
    \item{\textbf{Địa chỉ}} Đón tại Bưu điện Thành phố hoặc 92-96 Nguyễn Huệ
    \item{\textbf{Giờ mở cửa:}} 9 giờ đến 23 giờ 30 hàng ngày
    \item{\textbf{Giá vé/ chi phí:}} 300.000 đồng/người
    \item{\textbf{Phân loại điểm du lịch} }
    \item{\textbf{Nguồn thông tin}}
    \item{\textbf{Thông tin về địa điểm:}}
\end{itemize}

Địa điểm du lịch Sài Gòn thoải mái nhất chính là ở xe buýt 2 tầng. Bạn chỉ việc mua vé, chọn chỗ ngồi thích hợp và ngắm nhìn trọn các điểm tham quan mà xe buýt sẽ đưa bạn đến. Mỗi hành trình khám phá Sài Gòn từ xe buýt 2 tầng sẽ kéo dài khoảng 60 phút. Xe sẽ đi qua các địa điểm như Thảo Cầm Viên, bảo tàng Chứng tích chiến tranh, phố Tây Bùi Viện… Kết thúc hành trình xe buýt 2 tầng, bạn sẽ được ngắm thành phố từ trên cao tại tầng 49 của tòa tháp Saigon Skydeck.

\begin{itemize}
    \item{\textbf{Các thông tin khác - Cách di chuyển}}
\end{itemize}

\section{Tham quan Thảo Cầm Viên}
\begin{itemize}
    \item{\textbf{Địa chỉ}} 2 Nguyễn Bỉnh Khiêm, phường Bến Nghé, quận 1
    \item{\textbf{Giờ mở cửa:}} 7 – 17 giờ 30
    \item{\textbf{Giá vé/ chi phí:}} 60.000 đồng/người cao trên 1.3m, 40.000 đồng/người cao từ 1 – 1.3m, trẻ em dưới 1m hoàn toàn miễn phí
    \item{\textbf{Phân loại điểm du lịch} }
    \item{\textbf{Nguồn thông tin}}
    \item{\textbf{Thông tin về địa điểm:}}
\end{itemize}

Thảo Cầm Viên của Sài Gòn đã có tuổi đời hơn 150 năm. Đây là vườn thú có tuổi thọ thuộc top 8 trên thế giới. Tại Thảo Cầm Viên có đến 1.000 cá thể động vật và hơn 2.000 cây gỗ cùng hàng trăm loại cây thân thảo. Vườn Bách Thú không chỉ là nơi để tham quan mà còn là nơi nghiên cứu khoa học của các cá nhân, cơ quan tại Sài Gòn.

\begin{itemize}
    \item{\textbf{Các thông tin khác - Cách di chuyển}}
\end{itemize}

\section{Bảo tàng Lịch sử TP.HCM}
\begin{itemize}
    \item{\textbf{Địa chỉ}} 2 Nguyễn Bỉnh Khiêm, Phường Bến Nghé, Quận 1, TP. Hồ Chí Minh.
    \item{\textbf{Giờ mở cửa:}} 8:00 - 11:30 và 13:00 - 17:00 từ thứ Ba đến Chủ Nhật.
    \item{\textbf{Giá vé/ chi phí:}} 30.000VNĐ/lượt.
    \item{\textbf{Phân loại điểm du lịch} }
    \item{\textbf{Nguồn thông tin}}
    \item{\textbf{Thông tin về địa điểm:}}
\end{itemize}
Bảo tàng Lịch sử TP. Hồ Chí Minh là một công trình kiến trúc cổ do người Pháp xây dựng và hoàn thành vào năm 1929. Đây là nơi lưu giữ và trưng bày nhiều tư liệu và hiện vật quý giá, phản ánh quá trình phát triển của Việt Nam từ thời tiền sử đến hiện đại trên các phương diện lịch sử, văn hóa.

\begin{itemize}
    \item{\textbf{Các thông tin khác - Cách di chuyển}}
\end{itemize}

\section{Bảo tàng Phụ nữ Nam Bộ}
\begin{itemize}
    \item{\textbf{Địa chỉ}} 200 - 202 Võ Thị Sáu, Phường Võ Thị Sáu, Quận 3, TP. Hồ Chí Minh.
    \item{\textbf{Giờ mở cửa:}} 7:30 - 17:00 từ thứ Hai đến Chủ Nhật.
    \item{\textbf{Giá vé/ chi phí:}} Đang cập nhật.
    \item{\textbf{Phân loại điểm du lịch} }
    \item{\textbf{Nguồn thông tin}}
    \item{\textbf{Thông tin về địa điểm:}}
\end{itemize}
Bảo tàng Phụ nữ Nam Bộ (tiền thân là Nhà Truyền thống Phụ nữ Nam Bộ) khánh thành vào ngày 18/5/1990 và hiện nay đã trở thành một trong những địa điểm du lịch Sài Gòn hấp dẫn đối với du khách, đặc biệt là khách du lịch nước ngoài. Đây là nơi trưng bày các tư liệu, hiện vật chiến tranh cách mạng và hiện vật văn hóa về cuộc sống, những đóng góp của người phụ nữ Nam Bộ trong lịch sử, đặc biệt là trong cuộc chiến chống ngoại xâm. Bảo tàng có nhiều hiện vật phong phú từ trang phục truyền thống, đồ dùng hàng ngày đến các tư liệu, hình ảnh quý báu về người phụ nữ miền Nam xưa.

\begin{itemize}
    \item{\textbf{Các thông tin khác - Cách di chuyển}}
\end{itemize}

\section{Bến Nhà Rồng - Bảo tàng Hồ Chí Minh}
\begin{itemize}
    \item{\textbf{Địa chỉ}} Số 1 Nguyễn Tất Thành, Phường 12, Quận 4, TP. Hồ Chí Minh.
    \item{\textbf{Giờ mở cửa:}} 7:30 - 17:00 từ thứ Hai đến Chủ Nhật.
    \item{\textbf{Giá vé/ chi phí:}} Đang cập nhật.
    \item{\textbf{Phân loại điểm du lịch} }
    \item{\textbf{Nguồn thông tin}}
    \item{\textbf{Thông tin về địa điểm:}}
\end{itemize}
Bến Nhà Rồng là địa điểm lịch sử nổi tiếng, là nơi Bác Hồ ra đi tìm đường cứu nước vào ngày 5/6/1911 trên con tàu Amiral Latouche Tréville của Pháp. Nơi đây hiện nay là Bảo tàng Hồ Chí Minh - chi nhánh TP. Hồ Chí Minh. Bảo tàng có 7 phòng trưng bày nhiều hiện vật, hình ảnh về cuộc đời và sự nghiệp cách mạng của Chủ tịch Hồ Chí Minh, tình cảm của nhân dân miền Nam đối với vị lãnh tụ kính yêu của dân tộc cùng với các tài liệu lịch sử quan trọng khác. Giữa sân của bảo tàng là bức tượng “Nguyễn Tất Thành ra đi tìm đường cứu nước” do điêu khắc gia Phạm Mười thực hiện.

\begin{itemize}
    \item{\textbf{Các thông tin khác - Cách di chuyển}}
\end{itemize}

\section{Thảo Cầm Viên Sài Gòn}
\begin{itemize}
    \item{\textbf{Địa chỉ}} 2B Nguyễn Bỉnh Khiêm, Phường Bến Nghé, Quận 1, TP. Hồ Chí Minh.
    \item{\textbf{Giờ mở cửa:}} 7:00 - 17:30 hàng ngày.
    \item{\textbf{Giá vé/ chi phí:}} 40.000 - 60.000 VNĐ/người.
    \item{\textbf{Phân loại điểm du lịch} }
    \item{\textbf{Nguồn thông tin}}
    \item{\textbf{Thông tin về địa điểm:}}
\end{itemize}
Thảo Cầm Viên là một trong những khu bảo tồn động thực vật lớn nhất Sài Gòn, nơi bạn có thể khám phá và học hỏi về nhiều loài động, thực vật quý hiếm. Thảo Cầm Viên có nhiều khu vực khác nhau như khu vực động vật, khu vực thực vật và khu vực giải trí với nhiều hoạt động thú vị cho trẻ em và gia đình.

\begin{itemize}
    \item{\textbf{Các thông tin khác - Cách di chuyển}}
\end{itemize}

\section{Phố đi bộ Nguyễn Huệ}
\begin{itemize}
    \item{\textbf{Địa chỉ}} Đường Nguyễn Huệ, Quận 1, TP. Hồ Chí Minh.
    \item{\textbf{Giờ mở cửa:}} Cả ngày.
    \item{\textbf{Giá vé/ chi phí:}} Miễn phí.
    \item{\textbf{Phân loại điểm du lịch} }
    \item{\textbf{Nguồn thông tin}}
    \item{\textbf{Thông tin về địa điểm:}}
\end{itemize}
Phố đi bộ Nguyễn Huệ là nơi lý tưởng để tận hưởng không khí nhộn nhịp về đêm của Sài Gòn. Đây cũng là nơi tổ chức nhiều sự kiện, lễ hội lớn. Phố đi bộ Nguyễn Huệ có nhiều cửa hàng, quán cà phê và nhà hàng nên có thể nói đây là nơi lý tưởng để du khách thưởng thức ẩm thực và mua sắm kết hợp.

\begin{itemize}
    \item{\textbf{Các thông tin khác - Cách di chuyển}}
\end{itemize}

\section{Nhà thờ Đức Bà}
\begin{itemize}
    \item{\textbf{Địa chỉ}} 01 Công xã Paris, Phường Bến Nghé, Quận 1, TP. Hồ Chí Minh.
    \item{\textbf{Giờ mở cửa:}} 5:00 đến 20:00
    \item{\textbf{Giá vé/ chi phí:}} Miễn phí.
    \item{\textbf{Phân loại điểm du lịch} }
    \item{\textbf{Nguồn thông tin}}
    \item{\textbf{Thông tin về địa điểm:}}
\end{itemize}
Nhà thờ Đức Bà (tên gọi đầy đủ là Nhà thờ chính tòa Đức Bà Sài Gòn) là một công trình kiến trúc nổi bật với hai tháp chuông cao gần 57m. Công trình được xây dựng từ năm 1877 đến 1880 theo đồ bán của kiến trúc sư người Pháp J.Bourard. Nhà thờ Đức Bà Đây không chỉ là nơi thờ phụng mà còn mang đậm lối kiến trúc cổ của Pháp, đồng thời cũng là biểu tượng văn hóa và lịch sử của thành phố. Nhà thờ Đức Bà thường thu hút rất nhiều du khách, đặc biệt vào các dịp lễ Giáng Sinh và Phục Sinh khi nhà thờ tổ chức các nghi lễ đặc biệt.

\begin{itemize}
    \item{\textbf{Các thông tin khác - Cách di chuyển}}
\end{itemize}

\section{Dinh Độc Lập - Hội trường Thống Nhất}
\begin{itemize}
    \item{\textbf{Địa chỉ}} 135 Nam Kỳ Khởi Nghĩa, Phường Bến Thành, Quận 1, TP. Hồ Chí Minh.
    \item{\textbf{Giờ mở cửa:}} 8:00 - 16:30 từ thứ Hai đến Chủ Nhật.
    \item{\textbf{Giá vé/ chi phí:}} 10.000 - 40.000 VNĐ/người.
    \item{\textbf{Phân loại điểm du lịch} }
    \item{\textbf{Nguồn thông tin}}
    \item{\textbf{Thông tin về địa điểm:}}
\end{itemize}
Dinh Độc Lập là di tích quốc gia đặc biệt tại Sài Gòn và là địa điểm du lịch độc đáo, thu hút nhiều du khách trong và ngoài nước. Nơi đây không chỉ nổi tiếng bởi lối kiến trúc đẹp và đồ sộ mà còn là nơi ghi dấu sự kiện lịch sử quan trọng của nước ta. Bên trong Dinh Độc Lập, bạn có thể tham quan các phòng họp, phòng làm việc và nhiều hiện vật lịch sử quý giá.

\begin{itemize}
    \item{\textbf{Các thông tin khác - Cách di chuyển}}
\end{itemize}




\newpage
\section*{{Ẩm thực}}
\end{document}